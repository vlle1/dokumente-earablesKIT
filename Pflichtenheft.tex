\documentclass[a4paper,12pt]{article}
\usepackage{amssymb}
\usepackage{amsmath}
\usepackage[utf8]{inputenc} % Umlaute
\usepackage[ngerman]{babel} % Umlaute
\usepackage[T1]{fontenc}    % Umlaute
\usepackage[margin=2.5cm]{geometry}
\usepackage{booktabs}
\usepackage{lmodern}

% Notwendig für Links im Text
\usepackage{hyperref}

% glossar, see http://en.wikibooks.org/wiki/LaTeX/Glossary
% muss NACH hyperref geladen werden, sonst funktionieren die Links nicht
\usepackage{glossaries}

% Kompatibilität
\ifx\pdftexversion\undefined
\usepackage[dvips]{graphicx}
\else
\usepackage[pdftex]{graphicx}
\DeclareGraphicsRule{*}{mps}{*}{}
\fi

\makeglossaries

%%%%%%%%%%%%%%%%%%%%%%%%%%%%%%%%%%%%%%%%%%%%%%%%%%%%%%%%%%%%%%%%%%%%%%
% Variablen                                 						 %
%%%%%%%%%%%%%%%%%%%%%%%%%%%%%%%%%%%%%%%%%%%%%%%%%%%%%%%%%%%%%%%%%%%%%%
\newcommand{\authorName}{Tec O'Brain (Entwickler: David Höglinger, Jan Ettrich, Erwin Müller, Benedikt Rittner, Valentin Quapil)}
\newcommand{\auftraggeber}{Karlsruhe Institute of Technology (Teco)}
\newcommand{\auftragnehmer}{\authorName}
\newcommand{\projektName}{Pflichtenheft Earables}
\newcommand{\tags}{\authorName, Pflichtenheft, KIT, Informatik, PSE}
\newcommand{\glossarName}{Glossar}
\title{\projektName}
\date{\today}

%%%%%%%%%%%%%%%%%%%%%%%%%%%%%%%%%%%%%%%%%%%%%%%%%%%%%%%%%%%%%%%%%%%%%%
% PDF Meta information                                 				 %
%%%%%%%%%%%%%%%%%%%%%%%%%%%%%%%%%%%%%%%%%%%%%%%%%%%%%%%%%%%%%%%%%%%%%%
\hypersetup{
  pdfauthor   = {\authorName},
  pdfkeywords = {\tags},
  pdftitle    = {\projektName)}
}

%%%%%%%%%%%%%%%%%%%%%%%%%%%%%%%%%%%%%%%%%%%%%%%%%%%%%%%%%%%%%%%%%%%%%%
% Create a shorter version for tables. DO NOT CHANGE               	 %
%%%%%%%%%%%%%%%%%%%%%%%%%%%%%%%%%%%%%%%%%%%%%%%%%%%%%%%%%%%%%%%%%%%%%%
\newcommand\addrow[2]{#1 &#2\\ }

\newcommand\addheading[2]{#1 &#2\\ \hline}
\newcommand\tabularhead{\begin{tabular}{lp{13cm}}
\hline
}

\newcommand\addmulrow[2]{ \begin{minipage}[t][][t]{2.5cm}#1\end{minipage}%
   &\begin{minipage}[t][][t]{8cm}
    \begin{enumerate} #2   \end{enumerate}
    \end{minipage}\\ }

\newenvironment{usecase}{\tabularhead}
{\hline\end{tabular}}

\usepackage{microtype}
%%%%%%%%%%%%%%%%%%%%%%%%%%%%%%%%%%%%%%%%%%%%%%%%%%%%%%%%%%%%%%%%%%%%%%
% GLOSSARY ENTRIES                 	                              	 %
%%%%%%%%%%%%%%%%%%%%%%%%%%%%%%%%%%%%%%%%%%%%%%%%%%%%%%%%%%%%%%%%%%%%%%

\newglossaryentry{Echtzeit}{name=Echtzeit, description={Bereitstellen/Anzeigen von Daten mit einer durch die Verarbeitung bedingten Verzögerung von bis zu ca. 2 Sekunden zwischen dem Anfallen der (Roh-)Daten und der Ausgabe bzw. Visualisierung.}}
\newglossaryentry{Vorgang}{name=Vorgang, description={Als Vorgang wird in diesem Pflichtenheft bezeichnet, wenn ein Modus ausgewählt ist und Start gedrückt wurde. Der Vorgang endet mit dem Drücken von Stopp bzw. wird mit dem Wechseln des Modus.}}
\newglossaryentry{BLE}{name=BLE, description={Bluetooth Low Energy ist eine Technologie, die Teil des Industriestandards Bluetooth ist und eine energiesparende, kabellose Kommunikation zwischen Geräten in einer Entfernung von bis zu ca. 10 Metern ermöglicht.}}
\newglossaryentry{Wearable Computer}{name=Wearable Computer, description={Unter dem Begriff Wearable Computer versteht man Computersysteme, die am Körper, unter der Kleidung oder als Implantat unter der Haut getragen werden können.}}
\newglossaryentry{IMU}{name=6-Achsen IMU, description={Ein 6-Achsen IMU ist ein Beschleunigungssensor mit Gyroskop.}}
\newglossaryentry{GUI}{name=GUI, description={GUI ist die Abkürzung für den englischen Begriff \glqq graphical user interface\grqq . Sie ist die Schnittstelle zwischen Mensch und Maschine und ermöglicht dem Nutzer die Eingabe/Steuerung, der Maschine.}}
\newglossaryentry{Cross-Platform Bibliothek}{name=Cross-Platform Bibliothek, description={Eine Cross-Platform Bibliothek ist nichts weiter als eine Bibliothek die auf Rechnersystemen mit verschiedener Architektur laufen kann.}}
\newglossaryentry{Steuerungsparameter}{name=Steuerungsparameter, description={Unter Steuerungsparametern fassen wir die Länge der Verbindungsintervalle zwischen Kopfhöhrern und Handy sowie die Abtastrate und den Wertebereich von integriertem Gyroskop und Beschleunigungssensor zusammen.}}
\newglossaryentry{Schrittfrequenz}{name=Schrittfrequenz, description={Die Schrittfrequenz gibt an wie viele Schritte pro Zeiteinheit gemacht werden.}}
\newglossaryentry{Rohdaten}{name=Rohdaten, description={Als Rohdaten werden unverarbeitete Daten bezeichnet.}}
\newglossaryentry{TTS}{name=Text-To-Speech, description={Ein Text-to-Speech-System (TTS) (oder Vorleseautomat) wandelt Fließtext in eine akustische Sprachausgabe um. Dabei erfolgt diese auf Deutsch oder auf Englisch, abhängig davon, was als Sprache eingestellt ist.}}
\newglossaryentry{Earables}{name=Earables, description={Eine Zusammenschließung des Wortes Wearable und Earphone. Dabei handelt es sich um Kopfhörer, die mit Sensoren ausgestattet sind.}}
\newglossaryentry{Vorgangsdaten}{name=Vorgangsdaten, description={Daten, die bei der Ausführung eines Vorgangs gespeichert werden (z.B. Schritte, Sit-ups,\dots).}}


%%%%%%%%%%%%%%%%%%%%%%%%%%%%%%%%%%%%%%%%%%%%%%%%%%%%%%%%%%%%%%%%%%%%%%
% THE DOCUMENT BEGINS             	                              	 %
%%%%%%%%%%%%%%%%%%%%%%%%%%%%%%%%%%%%%%%%%%%%%%%%%%%%%%%%%%%%%%%%%%%%%%
\begin{document}
 \pagenumbering{roman}
 \begin{titlepage}
\maketitle
\thispagestyle{empty} % Keine Seitennummer

\begin{verbatim}






	




\end{verbatim}
\begin{center}
\includegraphics[width=0.5\textwidth]{EarablesBild.PNG}
\end{center}
\begin{verbatim}






\end{verbatim}


  \begin{tabular}[t]{p{4 cm}p{8 cm}}
	%%Projekt:       & \projektName \\[1.2ex]
	Auftraggeber:  & \auftraggeber\\[1.2ex]
	Auftragnehmer: & \auftragnehmer\\[1.2ex]
	Version:       & 1.1\\[1.2ex]
  \end{tabular}


%%\begin{tabular}[t]{|p{4 cm}|p{8 cm}|}
%%\hline
%%\textbf{Datum} & \textbf{Autor(en)} \\
%%\hline
%%\hline
%%\today & \authorName \\
%%\hline
%%\end{tabular}
\end{titlepage}
         % Deckblatt.tex laden und einfügen
 \setcounter{page}{2}
 \tableofcontents          % Inhaltsverzeichnis ausgeben
 \clearpage
 \pagenumbering{arabic}

\section{Einleitung}
\Gls{Earables} gehören zu den \Gls{Wearable Computer}, sie sind also nichts anderes als intelligente Kopfhörer. Je nachdem wie sie ausgestattet werden bringen sie die unterschiedlichsten Funktionen mit sich. Neben der klassischen Ausstattung von Lautsprechern, Mikrophon und \gls{BLE} besitzen sie in unserem Fall zusätzlich eine \Gls{IMU}. Mit ihrer Hilfe ist man in der Lage die Bewegungen des Nutzers zu erkennen und aufzuzeichnen, um mit den gewonnenen Messdaten  beispielsweise die \Gls{Schrittfrequenz} zu messen.  \Gls{Earables} bilden also eine Schnittstelle zwischen Mensch und Computer. Da sie kaum von normalen Bluetooth Kopfhörern zu unterscheiden sind, haben sie bereits jetzt eine hohe soziale Akzeptanz in der Gesellschaft erlangt, im Gegensatz zu beispielsweise Smartglasses. Mit der Entwicklung und Forschung dieser Art von \Gls{Earables} beschäftigt sich das globale eSense Projekt von Nokia Bell Labs in Zusammenarbeit mit dem TECO. Hier wird gemeinsam nach möglichen Anwendungsfällen der \Gls{Earables} gesucht, wodurch auch dieses Projekt initiiert wurde.
\section{Zielbestimmung}
Die Ziele dieses Projektes lassen sich in drei Bereiche gliedern:
\begin{enumerate}

  \item Es soll eine \Gls{Cross-Platform Bibliothek} (Android/IOS) für die \Gls{Earables} der Plattform eSense entwickelt werden mit der es möglich ist, die Messdaten des Gerätes aufzuzeichnen und verschiedene \Gls{Steuerungsparameter} zu verändern.
  
  \item Es soll ein Erweiterungsmodul entwickelt werden, mit dem die übermittelten  \Gls{Rohdaten} automatisch verarbeitet und ausgewertet werden, um so festzustellen ob der Nutzer gerade läuft oder steht.
  
  \item Es soll eine App mit einer \Gls{GUI} entwickelt werden, die über verschiedene Modi verfügt. Einer dieser Modi soll in der Lage sein anzuzeigen ob der Nutzer gerade läuft oder steht und die \gls{Schrittfrequenz} und die Anzahl der zurückgelegten Schritte anzeigen.

\end{enumerate}

\subsection{Musskriterien}

  \begin{itemize}
    \item Die \Gls{Cross-Platform Bibliothek} soll in der Lage sein die gesammelten Daten der \Gls{Earables} aufzuzeichnen
    \item Die \Gls{GUI} der App muss so benutzerfreundlich gestaltet werden, dass der Benutzer intuitiv weiß wie man die \Gls{Steuerungsparameter} der \Gls{Earables} verändert
    \item Die Cross-Plattform Bibliothek soll in der Lage sein die verschiedenen \Gls{Steuerungsparameter} zu verändern.
    \item Das Erweiterungsmodul soll erkennen ob der Nutzer gerade \glqq läuft\grqq{}oder \glqq steht\grqq.
    \item Das Erweiterungsmodul soll die \gls{Schrittfrequenz} und die Schrittanzahl ermitteln können.
    \item Die App soll anzeigen können ob der Nutzer gerade läuft oder steht.
    \item Die App soll die aktuelle Schrittfrequenz und die Anzahl der gelaufenen Schritte des Nutzers anzeigen.
    \item In der App wird die Anzahl der gelaufenen Schritte gespeichert.
  \end{itemize}
\subsection{Wunschkriterien}
  \begin{itemize}
    \item Die App soll die zurückgelegte Distanz anzeigen.
    \item Die App soll über drei weitere Modi verfügen.
      \begin{itemize}
        \item\text Im Modus \glqq Zählmodus\grqq{} zählt die App wie viele Liegestützen oder Sit-ups der Nutzer macht.
        \item\text  Im Modus \glqq Lauschen\&Agieren\grqq{}kann der Nutzer seinen eigenen Trainingsplan erstellen, welcher dann über Text to Speech ausgegeben wird.
        \item\text  Im Modus \glqq Musikmodus\grqq{} wird Musik abgespielt, wenn der Nutzer läuft und automatisch pausiert, wenn der Nutzer steht. Sobald der Nutzer weiter läuft wird die Musik auch wieder weiterlaufen.
      \end{itemize}
    \item\text Während der \glqq Laufmodus \grqq{} aktiv ist, wird die aktuelle Schrittanzahl und die zurückgelegte Distanz live angezeigt.
    \item\text Wenn kein Laufvorgang aktiv ist wird im Laufmodus die Daten (Schritte, Distanz) des letzten Datums angezeigt an dem trainiert wurde. %% in die NFA aufnehmen
    \item\text Die App wird standardmäßig auf der Sprache des Smartphone eingestellt sein. (Bei einem Deutschsprachigem Smartphone Deutsch, sonst Englisch). Die Sprache kann in den Einstellungen zwischen Deutsch und Englisch gewechselt werden. 
    \item\text Trainingsdaten werden automatisch gespeichert; Importieren und Exportieren soll möglich sein.
  \end{itemize}
  \subsection{Abgrenzungskriterien}
  \begin{itemize}
    \item Es werden keine \Gls{Rohdaten} längerfristig gespeichert.
    \item Die Musik passt sich nicht der \Gls{Schrittfrequenz} des Nutzers an.
    \item Im Modus \glqq Zählen\grqq{}wird davon ausgegangen, dass der Nutzer wirklich nur Liegestützen oder Sit-ups macht. Das bewusste Austricksen des Systems durch andere Bewegungen wird nicht behandelt.
    \item Es werden nur Daten auf dem Smartphone gespeichert und nicht auf den \Gls{Earables}
    \item Die Daten des Mikrophons werden weder aufgezeichnet noch ausgewertet.
  \end{itemize}

\section{Produkteinsatz}
Unser Produkt besteht aus drei Teilen (der Bibliothek, dazu das Erweiterungsmodul und der App), dementsprechend gibt es auch verschiedene Zielgruppen und Anwendungsbereiche:
  \subsection{Zielgruppe}
  \begin{itemize}
    \item\textsf{Bibliothek:} Softwareentwickler
    \item\textsf{Erweiterungsmodul:} Softwareentwickler
    \item\textsf{App:} Hobbysportler
  \end{itemize}
  \subsection{Anwendungsbereiche}
    \begin{itemize}
      \item\textsf{Bibliothek} \begin{itemize}
        \item[] Softwareentwicklung für eSense Wearables 
        \item[] %%! 
      \end{itemize}
      \item\textsf{Erweiterungsmodul}
      \begin{itemize}
        \item[] Softwareentwicklung im Bereich Schritterkennung.
        \item[] Auswerten der von \Gls{Earables} generierten Daten für die Verwendung in einer App z.B. im Bereich Fitness, aber auch als Nutzereingabe für andere interaktive Apps.
      \end{itemize}
      \item\textsf{App} Heimtraining, Sport,\dots
    \end{itemize}
  \subsection{Betriebsbedingungen der App} %% hier wollte David noch was hinzufügen
    \subsection{Physikalische Umgebung}
      Die physikalische Umgebung der Anwendung ist hardwareabhängig.
    \subsection{Sonstiges:}
    \begin{itemize}
      \item \textsf{Betriebsdauer:} \glqq Akkuabhängig\grqq, keine weitere Grenzen.
      \item \textsf{Qualifikation des Nutzers:} Sollte Sportübungen selbstständig ausführen können.
      \item \textsf{Vordergrund:} Die App muss im Vordergrund laufen, damit die Vorgänge funktionieren, siehe [/NF065/].
    \end{itemize}
      
\section{Produktumgebung}
Es handelt sich bei der App um eine Smartphoneanwendung, daher wird das Produkt als Installationspaket ausgeliefert. Außerdem setzen wir bestimmte Hardware- und Softwarebedingungen voraus.
\subsection{Hardware} 
	\textsf{Minimale Anforderungen:} Smartphone mit \Gls{BLE} Unterstützung.
	\textsf{Leistungsanforderung:} Das Smartphone sollte mindestens 1 gb RAM besitzen
	\textsf{Speicheranforderung:} Das Smartphone sollte mindestens 100 mb Speicherplatz zur Verfügung haben
\subsection{Software} \textsf{Betriebssystem:} Unterstützung nur von Android ab Version 7 und iOS ab Version 10.

\section{Funktionale Anforderungen}
Die Funktionalen Anforderungen gliedern sich in einen Pflichtteil (was unbedingt umgesetzt werden muss) und einen Wunschteil (Anforderungen, die umgesetzt werden sollen, aber nicht essenziell sind). Die Funktionalen Anforderungen spezifizieren auf Ebene der Bibliothek, des Erweiterungsmoduls und der App, welche Funktionen umgesetzt werden müssen bzw. können. 
  \subsection{Mussanforderungen}
    \subsubsection{Bibliothek}
    \begin{itemize}
      \item[/F010/] Daten des \Gls{IMU} in \Gls{Echtzeit} zur Verfügung stellen.
      \item[/F030/] Messparameter (Abtastrate, Wertebereich Gyroskop/Beschleunigungssensor, Tiefpassfilter) ändern. 
      \item[/F040/] Datenaufnahme des \Gls{IMU} starten und stoppen.
    \end{itemize}
    \subsubsection{Erweiterungsmodul}
     Auswertung der ausgelesenen Daten:
     \begin{itemize}
      \item[/F060/] \textsf{Schritterkennung} Es wird erkannt ob der Nutzer ein Schritt tätigt.
      \item[/F140/] \textsf{Schrittfrequenzerkennung} Während der Nutzer läuft wird die Schrittfrequenz ermittelt.
      \item[/F150/] \textsf{Schrittanzahl wird gezählt} Die Schritte des Nutzers werden gezählt.
    \end{itemize}
    \subsubsection{App}
      \begin{itemize}
      \item[/F070/] \textsf{App starten:} Der Nutzer kann die App über sein Smartphone starten. Die App startet im Laufmodus.
      \item[/F090/] \textsf{Modus wechseln:} Wechseln zwischen Modi über ein einblendbares Menü, während ein Modus ausgewählt ist.
      \item[/F100/] \textsf{Laufmodus:} Liveanzeige, ob Nutzer gerade \glqq läuft\grqq{} oder \glqq steht\grqq{} mit Hilfe von /F060/.
      \item[/F110/] \textsf{Vorgang starten:} Der Nutzer kann den modusspezifischen \Gls{Vorgang} starten. Dann wird der Modus nach seiner Beschreibung aktiv ausgeführt. Dies gilt für jeden Modus außer den Modus \glqq Livedaten\grqq.
      \item[/F120/] \textsf{Vorgang stoppen:} Der Nutzer kann den modusspezifischen \Gls{Vorgang} stoppen.
      \item[/F130/] \textsf{Resultat anzeigen:} Nach Stoppen Anzeigen des Vorgangsresultats bis neuer \Gls{Vorgang} gestartet wird.
      \item[/F132/] \textsf {Schrittfrequenz Anzeigen:} Es wird die Schrittfrequenz im Modus \glqq{}Laufmodus\grqq{} angezeigt, sobald der Modus gestartet wird. Die Schrittfrequenz wird mit Hilfe von /F140/ ermittelt.
      \item[/F134/] \textsf {Schrittanzahl Anzeigen:} Es wird die Schrittanzahl im Modus \glqq{}Laufmodus\grqq{} angezeigt, sobald der Modus gestartet wird. Die Schrittanzahl wird mit Hilfe von /F150/ ermittelt. Die Schrittzählung beginnt mit dem starten des Laufmodus.
      \item[/F136/] \textsf{Schrittanzahl Speichern:} Die App speichert die Schrittanzahl pro Tag und das für die letzten 30 Tage, an denen der Laufmodus aktiv war.
      \item[/F138/] \textsf{Ergebnisse der letzten Session anzeigen:} Wenn sich die App im Laufmodus befindet und der Laufmodus noch nicht gestartet ist wird angezeigt wie weit der Nutzer das letzte mal gelaufen ist und wie viele Schritte er gemacht hat. Falls es noch keine Einträge gibt wird nichts angezeigt.
    \end{itemize}
  \subsection{Wunschanforderungen}
    \subsubsection{Erweiterungsmodul}
      Weitere Datenauswertung:
      \begin{itemize}
      \item[/F170/] \textsf{Erkennung Sit-ups} Das Erweiterungsmodul erkennt wann der Nutzer Sit-ups macht.
      \item[/F180/] \textsf{Erkennung Liegestütze} Das Erweiterungsmodul erkennt wann der Nutzer Liegestütze macht.
      \end{itemize} 
    \subsubsection{App}
      Weitere Modi:
      \begin{itemize}
      \item[/F190/] \textsf{Modus Livedaten: \textit{(versteckt)}} Visualisieren der Sensorrohdaten als Graphen.
      \item[/F200/] \textsf{Zählmodus:} Zählen von Liegestützen oder Sit-ups mit Hilfe von /F170/ und /F180/. Das Ergebnis wird im Modus \glqq{} Zählmodus \grqq{} anzeigen.
      \item[/F210/] \textsf{Start/Stopp Musikmodus:} Musik stoppt wenn Nutzer stehen bleibt, läuft wenn der Nutzer läuft. Bei diesem Modus werden keine Resultate angezeigt.
      \item[/F220/]{
        Modus \glqq Lauschen\&Agieren\grqq
        \begin{itemize}
          \item[/F221/] Zusammenstellen eines Trainingsablaufs (Liegestütze, Sit-ups, Laufen). 
          \item[/F222/] Sprachanweisungen für die nächste Übung während des Trainings. 
          \item[/F223/] Anzeige der Zeitdauer jeder Übung nach Ablaufsende, siehe F130.
          \item[/F224/] Die Sprachanweisung erfolgt in der Sprache der in-App Einstellungen. (Deutsch oder Englisch)
        \end{itemize}
      }

      Einstellungen:
      \item[/F250/] Der Nutzer kann seinen Namen in den Einstellungen ändern.
      \item[/F260/] Der Nutzer kann die Sprache der App anpassen. (Möglichkeiten sind Deutsch und Englisch)
      \item[/F265/] Standardmäßig wird sich die Sprache der App automatisch an die Systemsprache des Smartphones des Nutzers anpassen.
      \item[/F270/] Der Nutzer kann die gespeicherten \Gls{Vorgangsdaten} löschen.
      \item[/F280/] Der Nutzer kann die \Gls{Steuerungsparameter} anpassen. 
      \item[/F285/] Der Nutzer kann die Schrittlänge für den Modus Distanzmessung anpassen.
      
      Sonstiges:
      \item[/F290/] Aufforderung der Angabe von Name und Schrittlänge bei Erstnutzung.
      \item[/F300/] Nutzer kann in der App seine gesamten \Gls{Vorgangsdaten} exportieren, importieren und löschen. Dabei wird das CSV-Format genutzt. 
      \item[/F310/] Speicherung der Summe der Schritte, Sit-ups und Liegestütze für jeden Tag, an dem ein \Gls{Vorgang} aktiv war (keine \Gls{Rohdaten}).
      \item[/F320/] Anzeigen der gespeicherten Daten aus /F310/.
      \end{itemize}


\section{Produktdaten}
Die App soll dem Nutzer die Möglichkeit zur Verfügung stellen, seine Trainingsdaten auch nach Beenden der App weiterhin zu Nutzen. Zu diesem Zweck werden die Trainingsdaten auf dem Smartphone gespeichert; immer sobald ein Vorgang abgeschlossen ist.
\begin{itemize}
	\item[/PD010/] Es werden keine rohen Messdaten gespeichert.
	\item[/PD020/] Die Einstellungen (Sprache, \Gls{Steuerungsparameter}, Benutzernamen) sind zu speichern. 
	\item[/PD040/] Die gesammelten \Gls{Vorgangsdaten} werden auf dem Gerät in einer Datenbank gespeichert, siehe /F310/.
\end{itemize}


\section{Nichtfunktionale Anforderungen}
Wie unten unter dem Punkt Qualitätszielbestimmungen aufgeführt legen wir unsere Priorität vor allem auf Korrektheit und Benutzerfreundlichkeit. Um die Benutzerfreundlichkeit zu gewährleisten, werden hier einige Anforderungen an die Stabilität, den Speicherplatz und das Laufzeitverhalten gestellt. Für das korrekte Funktionieren werden weitere allgemeine Anforderungen formuliert.
\subsection{Allgemein}
\begin{itemize}
  \item[/NF010/] Beim Ausführen der Funktion /F030/ soll die Datenaufnahme der \Gls{IMU} gestoppt werden.
  \item[/NF020/] Beim Wechseln zwischen Modi (/F090/) soll der aktuelle Modus terminiert werden.
  \item[/NF030/] Nach Ausführung der Funktion  Sprachänderung (/F260/) muss die App neu gestartet werden.
  \item[/NF040/] Bei nicht sinnvollen Angaben (z.B. negativen Werten in /F220/) wird das Starten des Vorgangs verhindert.
  \item[/NF050/] Bei Namensänderung (/F250/) werden gespeicherte Daten (siehe ev. /F300/) nicht verändert.
  \item[/NF060/] Bei Namensgebung sind nur Groß- und Kleinbuchstaben ohne Umlaute und Sonderzeichen erlaubt.
  \item[/NF065/] Gerät die App durch Minimieren oder Ausschalten des Smartphone Bildschirms in den Hintergrund, wird der aktuelle Vorgang terminiert.
  \item[] Handy und Kopfhörer sollten nicht über zehn Meter voneinander entfernt werden, sodass eine Bluetooth-Verbindung möglich ist. 
  \item[/NF067/] Beim Ausführen der Funktion /F070/ wird der Laufmodus nicht automatisch gestartet.
\end{itemize}
\subsection{Stabilität}
\begin{itemize}
  \item[/NF070/] Die App soll bei herkömmlicher Nutzung nicht öfter als zweimal bei zehnmaliger Benutzung. %stabilität
\end{itemize}
\subsection{Speicherplatz}
\begin{itemize}
  \item[/NF080/] Die App soll eine Größe von 100 MByte nicht überschreiten. %speicher
  \item[/NF090/] Die gespeicherten Daten sollen nicht mehr als 50 MByte umfassen. %speicher] 
\end{itemize}
\subsection{Laufzeitverhalten}
\begin{itemize}
  \item[/NF100/] Die Funktionen /F060/ und /F100/ (Laufmodus) soll maximal eine Verzögerung von zwei Sekunden aufweisen. %%Laufzeitverhalten
  \item[/NF110/] Der Start eines Modus nach seiner Auswahl (/F080/) soll nicht länger als zwei Sekunden benötigen. %%Laufzeitverhalten] 
  \item[/NF120/] Die Einblendung des Vorgangsresultats (/F130/) soll nicht länger als zwei Sekunden benötigen.%%Laufzeitverhalten

\end{itemize}
\section{Systemmodelle}
%%TODO Jan!!
  \subsection{Architekturdiagramm}
  \begin{center}
  	\vspace{100px}
  	\includegraphics[width=0.8\textwidth]{./Diagramme/Archi3.png}
  \end{center}
  
  \subsection{Use-Case-Diagramm}
  \begin{center}
	\includegraphics[width=0.8\textwidth]{./Diagramme/Use-CaseDiagramm.png} 
  \end{center}

Kurzbeschreibung: Beim Starten der App landet der Nutzer automatisch im Laufmodus. Bevor ein Modus genutzt werden kann muss der Nutzer eine BLE Verbindung über ein Pop-up Fenster mit den Kopfhörern herstellen. Der Nutzer kann nun entweder den Modus wechseln oder einen \Gls{Vorgang} starten. Nach Beendigung oder Stoppen eines Vorgangs wird das Vorgangsresultat angezeigt. Der Nutzer kann außerdem die Vorgangsdaten des letzten Monats einsehen und seine Daten importieren/exportieren. Das Ändern von Einstellungen und der damit verbundenen \Gls{Steuerungsparameter} der \Gls{Earables} ist ebenfalls möglich.
\section{Benutzeroberfläche}
Die \Gls{GUI} wird durch die App realisiert. Dabei kann man die einzelnen Modi über das Navigationsmenü auswählen. Jeder Modus hat zwei Seiten; eine Übersichtsseite und eine aktive Seite, welche während des aktiven \Gls{Vorgang} angezeigt wird. Für die weiteren Funktionen(Einstellungen, Dateienexport/-import) bietet die Applikation ebenfalls spezielle Seiten an.
Die \Gls{GUI} kann getestet werden mithilfe des Design-Tools \glqq Mock-up\grqq{} über den angegebenen Link in der Fußnote \footnote[1]{Mock-Up Prototyp: https://app.moqups.com/448nRiafse/view/page/a70dc1e41?ui=0}


\begin{figure}[ht!]
	\centering

	
		\begin{minipage}{0.4\textwidth}
			\includegraphics[width=4cm,height=9cm]{./Benutzeroberflaeche/Laufmodus.png}
			\caption{Laufmodus}
			\vspace{30px}
		\end{minipage}
			\hfill
		\begin{minipage}{0.4\textwidth}
			\includegraphics[width=4cm,height=9cm]{./Benutzeroberflaeche/Laufmodus_aktiv.png}
			\caption{aktiver Laufmodus}
			\vspace{30px}
		\end{minipage}
		\begin{minipage}{0.4\textwidth}
			\includegraphics[width=4cm,height=9cm]{./Benutzeroberflaeche/Musikmodus.png}
			\caption{Musikmodus}
		\end{minipage}
		\hfill
		\begin{minipage}{0.4\textwidth}
			\includegraphics[width=4cm,height=9cm]{./Benutzeroberflaeche/Musikmodus_aktiv.png}
			\caption{aktiver Musikmodus}
			
		\end{minipage}
\end{figure}

\begin{figure}[ht!]
	
	\centering
	\begin{minipage}{0.4\textwidth}
		\includegraphics[width=4cm,height=9cm]{./Benutzeroberflaeche/Lauschen_und_Agieren.png}
		\caption{Lauschen und Agieren}
		\vspace{30px}
	\end{minipage}
	\hfill
	\begin{minipage}{0.4\textwidth}
		\includegraphics[width=4cm,height=9cm]{./Benutzeroberflaeche/Lauschen_und_Agieren_aktiv.png}
		\caption{Lauschen und Agieren aktiv}
		\vspace{30px}
	\end{minipage}
	\begin{minipage}{0.4\textwidth}
		\includegraphics[width=4cm,height=9cm]{./Benutzeroberflaeche/Zaehlmodus.png}
		\caption{Zählmodus}		
	\end{minipage}
	\hfill
	\begin{minipage}{0.4\textwidth}	
		\includegraphics[width=4cm,height=9cm]{./Benutzeroberflaeche/Zaehlmodus_aktiv.png}
		\caption{Zählmodus aktiv}
		
	\end{minipage}
\end{figure}
\begin{figure}[ht!]
	\centering
	\begin{minipage}{0.4\textwidth}
		\includegraphics[width=4cm,height=9cm]{./Benutzeroberflaeche/Sidemenu.png}
		\caption{Navigationsmenü}
		\vspace{30px}
	\end{minipage}
	\hfill
	\begin{minipage}{0.4\textwidth}
		\includegraphics[width=4cm,height=9cm]{./Benutzeroberflaeche/Datenuebersicht.png}
		\caption{Datenübersicht}
		\vspace{30px}
	\end{minipage}
	\begin{minipage}{0.4\textwidth}
		\includegraphics[width=4cm,height=9cm]{./Benutzeroberflaeche/Import_Export.png}
		\caption{Import und Export}
	\end{minipage}
	\hfill
	\begin{minipage}{0.4\textwidth}
		\includegraphics[width=4cm,height=9cm]{./Benutzeroberflaeche/Settings.png}
		\caption{Einstellungen}
	\end{minipage}
\end{figure}
\newpage
\clearpage

\section{Qualitätszielbestimmungen}
Der Erste Bestandteil des Projekts ist die Bibliothek mit dem Erweiterungsmodul. Diese sollen als Open Source Projekt einer breiteren Entwicklergemeinschaft zur Verfügung gestellt werden, daher legen wir besonderen Wert auf die Korrektheit und Zuverlässigkeit.\\
Dagegen ist bei der App hautpsächlich die Benutzerfreundlichkeit wichtig, während die Robustheit und Zuverlässigkeit der App keine so große Rolle spielen. Generell ergeben sich insgesamt folgende Prioritäten:
\begin{tabular}[t]{|c|c|c|c|c|c|}
  \hline
  \textbf{Kriterium} & \textbf{Sehr Wichtig} & \textbf{Wichtig} & \textbf{Weniger Wichtig} & \textbf{Unwichtig}\\
  \hline
  \hline
  Korrektheit & x & & &\\ %%sehr
  \hline
  Zuverlässigkeit & & x & &\\ %%wichtig
  \hline
  Robustheit & & & x &\\  %%weniger
  \hline
  Effizienz & & x & &\\ %%wichtig
  \hline
  Benutzerfreundlichkeit & x & & &\\ %%sehr
  \hline
  Vertrauenswürdigkeit & & & x &\\ %%un
  \hline

\end{tabular}

\section{Globale Testfälle und Szenarien}
Die Testfälle sollen sicherstellen, dass alle funktionalen Anforderungen korrekt implementiert und umgesetzt wurden.
Die Nummerierung der globalen Testfälle richtet sich nach der Nummerierung der funktionalen Anforderungen.\\
Die Szenarien werden verwendet, um alle geforderten und optionalen Funktionen im Kontext einer Nutzerinteraktion zu testen.\\
Erst nach dem bestehen aller hier aufgeführten Tests und Szenarien ist das Produkt bereit für die Abnahme.
%%TODO David
  \subsection{Globale Testfälle}
  \subsubsection{Mussanforderungen}
  \paragraph{Bibliothek}
  \begin{itemize}
    \item[/T010/] Die Bibliothek kann die Daten des \Gls{IMU} in \Gls{Echtzeit} zur Verfügung stellen.
    \item[/T030/] Mithilfe der Bibliothek lassen sich (Abtastrate, Wertebereich Gyroskop/Beschleunigungssensor, Tiefpassfilter) ändern. 
    \item[/T040/] Die Bibliothek erlaubt es die Datenaufnahme des \Gls{IMU} zu starten und zu stoppen.
  \end{itemize}
  \paragraph{Erweiterungsmodul}
  \begin{itemize}
    \item[/T060/] Das Erweiterungsmodul erkennt Schritte.
    \item[/T140/] Ermittelt die aktuelle \Gls{Schrittfrequenz} korrekt.
    \item[/T150/] Zählt die Schritte des Nutzers.
  \end{itemize}
  \paragraph{App}
    \begin{itemize}
    \item[/T070/] Der Nutzer kann die App über sein Smartphone starten. Die App startet im Laufmodus.
    \item[/T090/] Der Nutzer kann die Modi über ein einblendebares Menü wechseln, während ein Modus ausgewählt ist.
    \item[/T100/] Der Nutzer kann sehen, ob er gerade \glqq läuft\grqq{} oder \glqq steht\grqq{}.
    \item[/T110/] Der Nutzer kann den modusspezifischen \Gls{Vorgang} starten. Dann wird der Modus nach seiner Beschreibung aktiv ausgeführt. Dies gilt für jeden Modus außer den Modus \glqq Livedaten\grqq.
    \item[/T120/] Der Nutzer kann den modusspezifischen \Gls{Vorgang} stoppen.
    \item[/T130/] Die App zeigt nach Stoppen des Vorgangs das Vorgansresultats an bis neuer \Gls{Vorgang} gestartet wird.
    \item[/T132/] Die App zeigt die \gls{Schrittfrequenz} im Modus \glqq{}Laufmodus\grqq{} an, sobald der Laufmodus gestartet wird.
    \item[/T134/] Die App zeigt die Schrittanzahl im Modus \glqq{}Laufmodus\grqq{} an, sobald der Laufmodus gestartet wird.
    \item[/T136/] Die App zeigt auf der Datenübersichtsseite die Anzahl der gelaufenen Schritte an.
    \item[/T138/] Die App befindet sich im Laufmodus und der Laufmodus ist noch nicht gestartet. Es wird angezeigt, wie viele Schritte der Nutzer bei der letzten Session gemacht hat und welche Strecke er dabei zurückgelegt hat.
  \end{itemize}

\subsubsection{Wunschanforderungen}
  \paragraph{Erweiterungsmodul}
    \begin{itemize}
    \item[/T170/] Erkennt Sit-ups. 
    \item[/T180/] Erkennt Liegestütze.
  \end{itemize}
  \paragraph{App}
  \begin{itemize}
    \item[/T190/] Die Sensorrohdaten werden als Graphen angezeigt.
    \item[/T200/] Die Liegestützen oder Sit-ups werden von der App gezählt und im Modus \glqq{}Zählmodus \grqq{} angezeigt.
    \item[/T210/] Musik stoppt wenn Nutzer stehen bleibt, läuft wenn der Nutzer läuft.
    \item[/T221/] Einen Trainingsplan zusammenstellen.
    \item[/T222/] Die App gibt dem Nutzer über Sprachanweisung (Deutsch) aus was die nächste Übung ist.
    \item[/T223/] Nach der Übung wird die benötigte Zeit angezeigt.
    \item[/T250/] Der Nutzer ändert seinen Namen.
    \item[/T260/] Der Nutzer ändert die Sprache auf Englisch.
    \item[/T270/] Trainingsdaten löschen.
    \item[/T280/] Die Samplingrate verändern.
    \item[/T285/] Die Schrittlänge verändern.
    \item[/T290/] Bei der Erstnutzung wird der Nutzer aufgefordert seinen Namen und seine Schrittlänge zu setzen.
    \item[/T300/] Trainingsdaten importieren, exportieren und löschen.
    \item[/T320/] Die gespeicherten Trainingsdaten ansehen.
   \end{itemize}
   
  \subsection{Szenarien}
    \subsubsection{Mussanforderungen}
      \paragraph{Laufmodus (/T100/)}
      \begin{enumerate}
        \item Das Smartphone wird per Bluetooth mit den Kopfhörern verbunden.
        \item Der Nutzer startet die App.
        \item Nach dem Startvorgang der App wird in den Modus \glqq Laufmodus\grqq{} gewechselt.
        \item Die Kopfhörer werden korrekt am Ohr des Nutzers angebracht.
        \item Die App zeigt dem Nutzer den Status \glqq stehend\grqq{} an.
        \item Sobald der Nutzer anfängt zu gehen zeigt die App \glqq gehend\grqq{} an.
        \item Sobald der Nutzer wieder still steht ändert sich der Zustand wieder zurück zu \glqq stehend\grqq. 
      \end{enumerate}

    \subsubsection{Wunschanforderungen}
        
    \paragraph{Livedaten (/F190/)}
      \begin{enumerate}
        \item Das Smartphone wird per Bluetooth mit den Kopfhörern verbunden.
        \item Der Nutzer startet die App. (/T070/)
        \item Nach dem Startvorgang der App wechselt der Nutzer in den Modus \glqq Livedaten\grqq . (/F190/)
        \item Die Kopfhörer werden korrekt am Ohr des Nutzers angebracht.
        \item Der Nutzer startet den gewählten \Gls{Vorgang}.
        \item Die App zeigt die korrekten Sensordaten an. (/T190/)
      \end{enumerate}

    
    \paragraph{Zählmodus (/F200/)}
      \begin{enumerate}
        \item Das Smartphone wird per Bluetooth mit den Kopfhörern verbunden.
        \item Der Nutzer startet die App. (/T070/)
        \item Nach dem Startvorgang der App wechselt der Nutzer in den Modus \glqq Zählmodus\grqq . (/F200/)
        \item Die Kopfhörer werden korrekt am Ohr des Nutzers angebracht.
        \item Der Nutzer wählt eine verfügbare Übung aus. 
        \item Der Nutzer startet den gewählten \Gls{Vorgang}.
        \item Der Nutzer führt die Übung X (natürliche Zahl) mal aus.
        \item Der Nutzer stoppt den laufenden \Gls{Vorgang}.
        \item Die App zeigt X an. (/T200/)
      \end{enumerate}

    
    \paragraph{Start/Stop Musikmodus (/F210/)}
      \begin{enumerate}
        \item Das Smartphone wird per Bluetooth mit den Kopfhörern verbunden.
        \item Der Nutzer spielt Musik mit der vorinstallierten Musik-App ab.
        \item Der Nutzer startet die App. (/T070/)
        \item Nach dem Startvorgang der App wechselt der Nutzer in den Modus \glqq Laufmodus\grqq . (/F210/)
        \item Die Kopfhörer werden korrekt am Ohr des Nutzers angebracht.
        \item Der Nutzer beginnt zu gehen.
        \item Der Nutzer startet den gewählten \Gls{Vorgang}.
        \item Der Nutzer hört auf zu gehen.
        \item Die Musik wird pausiert. (/T210/)
        \item Der Nutzer geht weiter.
        \item Die Musik startet automatisch wieder. (/T210/)
      \end{enumerate}
  
      \paragraph{Lauschen\&Agieren (/F220/)}
      \begin{enumerate}
        \item Das Smartphone wird per Bluetooth mit den Kopfhörern verbunden.
        \item Der Nutzer startet die App. (/T070/)
        \item Nach dem Startvorgang der App wechselt der Nutzer in den Modus \glqq Lauschen\&Agieren\grqq . (/F220/)
        \item Die Kopfhörer werden korrekt am Ohr des Nutzers angebracht.
        \item Der Nutzer stellt sich ein Training aus den verfügbaren Übungen zusammen (/T221/).
        \item Der Nutzer startet den gewählten \Gls{Vorgang}.
        \item Dem Nutzer wird per \Gls{TTS} die aktuelle Übung angesagt (/T222/).
        \item Der Nutzer führt die Übung aus.
        \item Die letzten beiden Schritte werden so lange wiederholt, bis alle ausgewählten Übungen erledigt sind.
        \item Der \Gls{Vorgang} wird automatisch beendet.
        \item Die App zeigt die benötigten Zeiten pro Übung an (/T223/).
      \end{enumerate}

      \paragraph{Einstellungen (/F250/ bis /F285/)}
      \textit{Voraussetzung: /T220/}
      \begin{enumerate}
        \item Das Smartphone wird per Bluetooth mit den Kopfhörern verbunden.
        \item Der Nutzer startet die App. (/T070/)
        \item Nach dem Startvorgang der App wechselt der Nutzer in die Ansicht \glqq Einstellungen\grqq .
        \item Der Nutzer ändert seinen Namen (/T250/).
        \item Der Nutzer ändert die Sprache auf Englisch (/T260/).
        \item Dem Nutzer löscht die gespeicherten \Gls{Vorgangsdaten} (/T270/).
        \item Dem Nutzer ändert die \Gls{Steuerungsparameter} (/T280/).
        \item Der Nutzer ändert seine Schrittlänge (/T285/).
        \item Der Nutzer beendet die App.
        \item Der Nutzer startet die App. (/T070/)
        \item Der Nutzer wechselt in die Ansicht \glqq Einstellungen\grqq .
        \item Die Einstellungen sind exakt so, wie sie vorher eingestellt wurden.
      \end{enumerate}

      \paragraph{Erstnutzung (/F290/)}
      \begin{enumerate}
        \item Der Nutzer startet die App zum ersten Mal. (/T070/)
        \item Der Nutzer wird aufgefordert seinen Namen und seine Schrittlänge zu setzen. (/T290/)
        \item Der Nutzer ändert die Sprache auf Englisch.
        \item Danach wechselt die App in den \glqq Laufmodus\grqq.
        \item Der Nutzer schließt die App.
        \item Der Nutzer startet die App erneut und befindet sich nun im \glqq Laufmodus\grqq. (/T070/)
        \item Sein Name, seine Schrittlänge und die in-App Sprache wurden gespeichert.
      \end{enumerate}

      \paragraph{Vorgangsdaten importieren/exportieren (/F300/)}
      \begin{enumerate}
        \item Der Nutzer startet die App. (/T070/)
        \item Der Nutzer wählt über das Hamburgermenü den Reiter \glqq{}Import/Export \grqq{}.
        \item Der Nutzer exportiert seine Daten. (/T300/)
        \item Der Nutzer löscht seine Daten. (/T300/)
        \item Der Nutzer importiert seine Daten. (/T300/)
        \item Die Daten sind alle wieder vorhanden so als ob der Nutzer sie nie gelöscht hätte.
      \end{enumerate}

\section{Entwicklungsumgebung}
Wir arbeiten an dem Projekt mit Visual Studio 2017/2019, sodass alle eine einheitliche Entwicklungsumgebung verwenden.\\
Dabei kommt der .NET Standard 2.0 zum Einsatz, der C\# 7.2 verwendet. Wir arbeiten außerdem mit Xamarin Forms.\\
Zur Versionskontrolle und zur Projektübersicht wird Git verwendet, das Repository liegt öffentlich auf Github\footnote{\url{https://github.com/vlle1/earablesKIT}}.
\clearpage
\printglossaries
\stepcounter{section}
\end{document}

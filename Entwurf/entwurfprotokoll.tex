\documentclass[a4paper,12pt]{article}
\author {ab Protokoll 12.3.19}
\begin{document}
\section{Klassenstruktur}

\subsection{Model}
\subsubsection{Dangling}
- Fassade, die Zugriff auf Services und auf Bibliothek Zugriff hat
- Dependency Sercvices kann von außen aufgerufen werden (``global'', von Xamarin), verwaltet Singletons
Folgende Services gibt es:
\begin{itemize}
    \item Bibliothek
    \item Datenkodul
    \item Erweiterungsmodul
    \item Datenmodul
\end{itemize}
\subsubsection{Bibliothek} 
Bis jetzt unstrukturiert, schnittstellen noch unklar...
Bestandteile der Bibliothek:
%%was die Bibliothek braucht:
\begin{itemize}
    \item {
        get/set Parameter (konstanten)
        \begin{itemize}
            \item Samplingrate
            \item Accelerator/Gyroscope(range/filter)
        \end{itemize}
        }
    \item Daten aufzeichnen %%!!
    \item Start/Stop Samplingrate
    \item {
        Daten verpacken in Format per push (Event) (Beobachter)\\
        Wann? %%informal
        \begin{itemize}
            \item bei ca. 100 Rohdaten
            \item configs %%??
        \end{itemize}
        } 
\end{itemize}

\subsubsection{Erweiterungsmodul}
Nach außen:\\
\textbf{async registerEvent(), unregisterEvent()}\\
mit Beobachter: man kann sich Registrieren für Schritterkennung, Liegest. und Situpsevents
Verarbeitung, wenn mindestens ein Listener registriert ist. 

Methoden
\begin{itemize}
    \item getStepFrequency()
    \item getStepAmount()
\end{itemize}
%%Schritterkennung/ Situpserkennung / Liegestütze ??
Events:
\begin{itemize}
    \item onWalkingStateChanged(WalkingEvent: e)
    \item onActivityUpdate(ActivityEvent: e) // 
\end{itemize}
Enum ActivityEvent: WalkingStateEvent, PushupEvent, SitupEvent

\subsection{Viewmodel}

\subsection{View}


\section{Schnittstellen zwischen Model, Viewmodel und View}
Verbindung zwischen Bibliothek und Erweiterungsmodul:
\end{document}
 